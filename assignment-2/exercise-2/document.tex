% !TEX root = ./document.tex

\documentclass[a4paper, spanish]{article}

\usepackage{mystyle}
\usepackage{myvars}

\begin{document}

  \maketitle

  \begin{itemize}
    \item \textbf{Archivo}: \texttt{tuberculo.sf3}
    \item \textbf{Serie}: Número de casos registrados semanalmente de tuberculosis respiratoria en España, entre los años $1982$ y $1991$ (el primer dato corresponde al número de casos registrados desde el \emph{$1$ de Enero de $1982$} al \emph{$7$ de Enero de $1982$}).
    \begin{itemize}
      \item $\{X_t\}$ Serie Original.
      \item $\{Y_t\}$ Serie del número de casos en periodos de cuatro semanas sucesivos.
    \end{itemize}
  \end{itemize}

  \section{Describir estas dos series ($\{Y_t\}$ puede crearse con el \texttt{proc expand} de \emph{SAS}), indicando claramente para cada una de ellas qué frecuencias elegiríais a priori para ajustar un modelo con tendencia polinómica más ondas.}
  \label{sec:a}

    \paragraph{}
    En este trabajo se va a analizar una serie temporal unidimensional referida al número de casos registrados en España de problemas re tuberculosis respiratoria. La serie original se da en periodos acumulados de manera semanal, por lo que el número de observaciones anuales es variable de $52$. Sin embargo, nótese que este valor puede variar sutilmente debido a los días concretos de comienzo y fin de cada año.

    \paragraph{}
    Otra de las dificultades de este análisis procede de la enorme amplitud del periodo de la serie, que complica el análisis estacional, así como el ajuste de la serie a nivel computacional drásticamente. Por esta razón, se va a proceder al estudio de la serie mediante agrupamiento de las observaciones en periodos de \emph{4 semanas}. Esto implica $52 / 4 = 13$ observaciones anuales, las cuales se pueden analizar de una manera más cómoda sin perder demasiada información.

    \paragraph{}
    Para el estudio de las series en cuestión (así como su ajuste) se utilizará el sistema \texttt{SAS}, para el cual se incluyen distintos fragmentos de código que permiten la reconstrucción del análisis. La mayoría de dichos bloques de código se han incluido en el apartado en cuestión del documento. Sin embargo, otros que no se han considerado lo suficientemente relevantes para el estudio se han incluido en el Apéndice \ref{appendix:source_code_sas}. Para la generación de algunos gráficos, así como otras partes del análisis, se ha utilizado el lenguaje \texttt{R}, cuyo código fuente se incluye en el Apéndice \ref{appendix:source_code_r} por la misma razón que el caso anterior.

    \paragraph{}
    El resto del apartado se divide de la siguiente manera: en el \autoref{sec:weekly_analysis} se estudia la serie sin agrupar, para posteriormente realizar el agrupamiento y analizar la serie resultante en \autoref{sec:weekly4_analysis}.

    \subsection{Serie Semanal: $\{X_t\}$}
    \label{sec:weekly_analysis}

      \paragraph{}
      Para el análisis de la serie semanal, a la cual denotaremos por $\{X_t\}$ de forma matemática, lo primero es cargarla en el sistema \texttt{SAS}, lo cual se puede llevar a cabo a través del la ejecución del \autoref{code:data_import}. Lo primero en que nos fijamos es el número de observaciones de la serie, las cuales son $520$ ($52$ observaciones y un total de $10$ años). Otra de las cuestiones que hay que tener en cuenta para que el análisis de la serie sea realizado de manera adecuada es fijar la fecha de manera coherente. Si recurrimos al calendario, el \emph{1 de Enero de 1982} fue \emph{Domingo}, lo cual debemos indicar para que las fechas sean construidas de manera adecuada dado que en las observaciones frontera entre un distintos años, esto puede influir de manera perjudicial para el análisis.

      \begin{figure}[htb!]
        \centering
        \includegraphics[width=\linewidth]{semanal}
        \caption{Gráfico de la serie, gráfico de dispersión \emph{rango-media}, \emph{correlograma} y \emph{periodograma} del conjunto de datos \texttt{EJ2.SEMANAL}}
        \label{img:semanal}
      \end{figure}

      \paragraph{}
      Una vez cargada formateada de manera apropiada la fecha en la serie, ya se puede proceder al análisis descriptivo de la misma. Para ello se ha utilizado el \autoref{code:semanal_analysis}, a través del cual hemos obtenido la \autoref{img:semanal}. Tal y como se puede apreciar a través del gráfico de la serie, esta presenta una tendencia que podría aproximarse a partir de un polinomio de orden cuadrático, alcanzando su máximo entre los años $1986$ y $1987$. Algo interesante es la forma de dicha tendencia, que experimenta un crecimiento de forma lineal desde $1981$ hasta su máximo, para después sufrir un fuerte decrecimiento (que también se podría considerar lineal) durante el año $1987$. El resto de la serie ($[1988,1992)$) mantiene una tendencia que podría ser considerada constante. Por lo tanto, esta serie presenta cambios de tendencia (lo cual requerirá modelos basados en \emph{Suavizado Exponencial} tal y como veremos en los próximos apartados). Además de esto, el gráfico de la serie parece mostrar una componente estacional, sin embargo es dificil ver esto, por lo que posteriormente procederemos a diferenciarla para tratar de visualizarlo de manera más clara. Sin embargo, tal y como se indica a continuación, esto se aprecia mejor en la serie agrupada cada cuatro semanas $\{Y_t\}$.

      \paragraph{}
      Continuando con los diagramas de la \autoref{img:semanal}, a partir del gráfico de dispersión \emph{rango-media}, se puede apreciar cierta relación entre estas, lo cual implica una relación entre el nivel y la dispersión en la serie. Es decir, cuanto mayor es el valor medio en la serie, mayor es la varianza de los datos. Para corregir esto existen distintos métodos, entre los que se encuentran transformaciones de tipo \emph{Box-Cox} que tratan de eliminar dicha relación. Sin embargo, el estudio de dichos métodos no son el objetivo de este análisis.

      \paragraph{}
      En cuanto al \emph{correlograma}, en este se aprecia una alta correlacion entre las variables (lo cual indica tendencia). Estos decrecimientos parecen localmente exponenciales (lo cual indicaría estacionaridad local), por lo que modelos basados en suavizado exponencial deberían ajustarse bien a la serie. A partir de este gráficos, podemos intuir la estructura estacional de la serie. Sin embargo, es complicado poder estudiarla detalladamente debido a la tendencia, así como el enorme periodo de la serie. Aunque destaca sobre el resto la corelación $52$ como era de esperar, ya que indica el periodo de la serie. También son llamativas las próximas a las correlaciones $17$ y $34$, lo cual podría indicar un periodo \emph{cuatrimestral}.

      \paragraph{}
      Para finalizar, en el periodograma se aprecia que el primer armónico destaca fuertemente sobre el resto, lo cual se debe a la tendencia de la serie. Debido al elevado valor que toma dicho armónico, no es posible estudiar el resto de valores de manera sencilla.

      \paragraph{}
      Para tratar de reducir la componente de tendencia y poder analizar de manera más cómoda la estacionalidad, procedemos a realizar una diferenciación sobre la serie. En la \autoref{img:semanal_diff} se muestran los gráficos descriptivos referidos a la serie diferenciada.

      \begin{figure}[htb!]
        \centering
        \includegraphics[width=\linewidth]{semanal-diff}
        \caption{Gráfico de la serie, gráfico de dispersión \emph{rango-media}, \emph{correlograma} y \emph{periodograma} del conjunto de datos \texttt{EJ2.SEMANAL} tras aplicar una diferenciación de primer orden.}
        \label{img:semanal_diff}
      \end{figure}

      \paragraph{}
      Lo primero que llama la atención tras la diferenciación es la eliminación completa de la tendencia. En las zonas de crecimiento y decrecimiento ahora se muestra un mayor grado de variabilidad. También se aprecia en mayor medida la estructura estacional, pero nuevamente el enorme periodo de la serie no permite un análisis sencillo sobre la misma (en el \autoref{sec:weekly4_analysis}) se puede estudiar más cómodamente la estacionalidad.

      \paragraph{}
      Siguiendo con la interpretación de los gráficos de la \ref{img:semanal_diff}, la relación entre nivel y dispersión se ha reducido drásticamente, lo cual se puede comprobar a través del gráfico \emph{rango-media}. En cuanto al \emph{correlograma}, este muestra una alta correlación en sentido negativo con el primer retardo. Sin embargo, lo que más destaca es la estructura cíclica del resto de correlaciones, la cual se puede apreciar mucho mejor en la serie agrupada que comentaremos posteriormente. Para finalizar, en el \emph{periodograma} se puede apreciar que hay armónicos que destacan sobre el resto y marcan tendencia, pero que estudiaremos en la serie agrupada cada cuatro semanas por comodidad y sencillez en el análisis.

    \subsection{Serie 4 semanas: $\{Y_t\}$}
    \label{sec:weekly4_analysis}

      \paragraph{}
      Para trabajar con la serie agrupada en periodos de cuatro semanas. Es decir, la suma de observaciones de de $4$ en $4$ se ha utilizado el \autoref{code:a_expand}. Este se basa en la sentencia \texttt{proc expand} que facilita el agrupamiento/separación de observaciones indexadas sobre un soporte temporal.

      \begin{listing}[htb!]
        \centering
        \inputminted{SAS}{./res/code/a-02-expand.sas}
        \caption{Generación del conjunto de datos \texttt{EJ2.SEMANAL4} a partir de \texttt{EJ2.SEMANAL}}
        \label{code:a_expand}
      \end{listing}

      \paragraph{}
      Una vez creado el conjunto de datos que se utilizará para el análisis, el siguiente paso es realizar un estudio descriptivo sobre el mismo. En este caso estamos trabajando con $520 / 4 = 130$ observaciones en periodos de $4$ semanas, comenzando el \emph{1 de Enero de 1982} que, como indicamos anteriormente, fue \emph{Domingo}. Es de especial importancia darse cuenta de que los periodos de $4$ semanas no coinciden con periodos mensuales. En este caso la amplitud del periodo es $13$, mientras que en los análisis mensuales, es de $12$ observaciones. Es necesario tener esto presente a lo largo del documento.

      \begin{figure}[htb!]
        \centering
        \includegraphics[width=\linewidth]{semanal4}
        \caption{Gráfico de la serie, gráfico de dispersión \emph{rango-media}, \emph{correlograma} y \emph{periodograma} del conjunto de datos \texttt{EJ2.SEMANAL4}}
        \label{img:semanal4}
      \end{figure}

      \paragraph{}
      En la \autoref{img:semanal4} se muestran distintos gráficos descriptivos referidos a la serie agrupada cada 4 semanas $\{Y_t\}$, los cuales procedemos a analizar e interpretar a continuación. Lo más destacado (y obvio) de estos es que son un reflejo simplificado de los de la serie semanal $\{X_t\}$ que se muestran en la \autoref{img:semanal}.

      \paragraph{}
      Si nos fijamos en el gráfico de la serie, seguimos viendo la tendencia, que indicamos anteriormente, pero en este caso de una manera mucho más sencilla de interpretar. Ahora se ve más clara la estructura cuadrática de la misma, junto con una relación entre nivel y dispersión (que se confirma sobre el gráfico de dispersión \emph{rango-media}). Otra de las partes que más llama la atención al visualizar el gráfico de la serie es que ahora es mucho más sencillo fijarse en la coponente estacional de la serie, que sigue una estructura periodica tras cada $13$ observaciones.

      \paragraph{}
      A partir del \emph{correlograma} se puede apreciar la misma estructura de correlaciones que se indicaba sobre la serie semanal, donde destaca una estructura periódica (que estudiaremos más en detalle sobre la serie diferenciada). Al igual que antes, la serie no puede ser considerada estacionaria porque el decrecimiento entre correlaciones no es exponencial, aunque de la misma manera que antes, si que presenta esta estructura de manera local, lo cual nos confirma que un ajuste basado en suavizado exponencial debería ser acertado (sin olvidar la componente estacional).

      \paragraph{}
      En cuanto al \emph{periodograma}, al igual que antes, uno de los primeros armónicos destacan fuertemente sobre el resto, lo cual intuimos que se debe a la tendencia de la serie. Además, en este caso se puede apreciar algo que no veíamos en la serie sin agrupar. Esto es que los armónicos de periodo $i/13$ destacan sutilmente sobre su entorno. En concreto, destacan sobre el resto los armónicos $1/13$ y $3/13$. Aunque el primero de ellos no parece hacerlo de manera determinista, ya que no coincide exactamente con el valor $1/13$, sino que es algo menor.

      \paragraph{}
      Para poder estudiar la componente estacional con mayor precisión se ha procedido a la diferenciación de la serie agrupada $\{Y_t\}$, cuyos gráficos descriptivos se muestran en la \autoref{img:semanal4_diff}. Al igual que antes, esta serie es una versión simplicada de la estudiada para la serie semanal $\{X_t\}$, que se muestra en la \autoref{img:semanal_diff}.

      \begin{figure}[htb!]
        \centering
        \includegraphics[width=\linewidth]{semanal4-diff}
        \caption{Gráfico de la serie, gráfico de dispersión \emph{rango-media}, \emph{correlograma} y \emph{periodograma} del conjunto de datos \texttt{EJ2.SEMANAL4} tras aplicar una diferenciación de primer orden.}
        \label{img:semanal4_diff}
      \end{figure}

      \paragraph{}
      El \emph{gráfico de la serie} repite de manera cíclica el mismo patrón, lo cual confirma la existencia de estacionalidad, aunque a través de este diagrama es difícil de interpretar. Al igual que antes, el diagrama de dispersión \emph{rango-media} muestra una estavilización de la dispersión, lo cual podría indicarnos que modelos basados en diferenciación no requerirán de una estabilización de la varianza previa mediante transformaciones. En cuanto al \emph{correlograma}, ahora es trivial interpretar la existencia de estacionalidad, debido a la estructura de correlaciones, que se repite cíclicamente alternando entre valores positivos y negativos. En cuanto al \emph{periodograma}, este nos muestra que los armónicos que siguen la forma $i/13$ destacan sobre el resto.

      \paragraph{}
      En el título del apartado se dice que es necesario indicar los armónicos significativos para la serie agrupada en periodos de $4$ semanas y que denotamos por $Y_t$. Por tanto, tras el análisis de la misma, podemos confirmar que el \textbf{armónico referido a la frecuencia $3/13$ es muy significativo}. También lo son, aunque de manera mas sutil los armónicos $1/13$, $4/13$, $6/13$ y $2 / 13$ respectivamente. Aunque para confirmar su significancia se deberían los respectivos contrastes de hipótesis.

    \paragraph{}
    Antes de finalizar el apartado, se concluye que el agrupamiento de la serie semanal original en observaciones de $4$ semanas es una estrategia que permite analizar la estructura inherente de la serie de manera mucho sencilla e intuitiva. Dado que el objetivo es tratar de comprender la estructura de la serie para así poder ajustar un modelo coherente para la misma, esto simplifica en gran medida el trabajo, no solo a nivel analítico, sino también a nivel computacional, dado que el número de parámetros a estimar se reduce drásticamente.

  \section{Ajustar por suavizado exponencial, con el \texttt{proc esm}, los tres modelos que se consideren más apropiados para la serie $\{Y_t\}$ y comprobar su adecuación.}
  \label{sec:b}

    \paragraph{}
    Tras realizar un exhaustivo análisis de todas las diferentes posibilidades de ajuste por suavizado exponencial elegimos el \textit{seasonal exponential smoothing}, el \textit{suavizado de Winters multiplicativo} y el \textit{aditivo}, ya que son los tres que incorporarán la estacionalidad de la serie.

    \subsection{Suavizado Exponencial con Estacionalidad}

      \begin{listing}[htb!]
        \centering
        \inputminted{SAS}{./res/code/b-01-esm-seasonal.sas}
        \caption{Código fuente para el ajuste de un modelo de \emph{Suavizado Exponencial con Estacionalidad} sobre el conjunto de datos \texttt{EJ2.SEMANAL4}}
        \label{code:b_seasonal_esm}
      \end{listing}

      \paragraph{}
      Para comenzar explicaremos el modelo SES en el que mediante la tabla de los estimadores del parámetro y la significancia de dicho test.

      \begin{table}[htb!]
        \centering
        \includegraphics[width=0.75\textwidth]{PvalorSeasonal}
        \caption{Significancia para el modelo de \emph{Suavizado Exponencial con Estacionalidad} sobre el conjunto de datos \texttt{EJ2.SEMANAL4}}
        \label{table:b_seasonal_significance}
      \end{table}

      \paragraph{}
      Observamos que para la constante de suavizado estacional no se rechaza su significancia a cualquier nivel ya que el $p-valor$ del test se encuentra próximo $0.99$.

      \paragraph{}
      En cambio, si se rechazará para la constante de suavizado para la media, donde $p-valor < 0.001$. Será conveniente diferenciar ya que el estimador estacional es cercano a 0. A continuación adjuntamos el gráfico del ACF de residuales:

      \begin{figure}[htb!]
        \centering
        \includegraphics[width=0.75\textwidth]{ErrorACFPlot_SEASONAL}
        \caption{Gráfico de autocorrelaciones (correlograma) para los residuales del modelo de \emph{Suavizado Exponencial con Estacionalidad} sobre el conjunto de datos \texttt{EJ2.SEMANAL4}}
        \label{img:b_seasonal_residuals_correlogram}
      \end{figure}

      \paragraph{}
      Observamos como dicho modelo ofrece dudas sobre su validación ya que para el retardo 2 la autocorrelación es muy alta. El retardo 1 no es muy alto, no resultará perjudicial para el modelo. Los retardos estacionales(cada 13) no son notables por lo que es bueno para la validación.

      \begin{figure}[htb!]
        \centering
        \includegraphics[width=0.75\textwidth]{ErrorWhiteNoiseLogProbPlot_SEASONAL}
        \caption{Gráfico sobre el contraste de ruido blanco para los residuales del modelo de \emph{Suavizado Exponencial con Estacionalidad} sobre el conjunto de datos \texttt{EJ2.SEMANAL4}}
        \label{img:b_seasonal_test_white_noise}
      \end{figure}

      \paragraph{}
      Vemos en la gráfica que el test de que las correlaciones sean cero se rechaza para los 4 primeros retardos a nivel 0.05. Esto es indeseable para validar el modelo puesto que no podemos asegurar que sea ruido blanco, que es lo que se busca.

      \paragraph{}
      A continuación, pasamos a realizar el ajuste del modelo aditivo de Winter.

    \subsection{Winter Aditivo}

      \begin{listing}[htb!]
        \centering
        \inputminted{SAS}{./res/code/b-01-esm-winteradd.sas}
        \caption{Ajuste de un modelo de \emph{Winter Aditivo} sobre el conjunto de datos \texttt{EJ2.SEMANAL4}}
        \label{code:b_winter_additive_esm}
      \end{listing}

      \begin{table}[htb!]
        \centering
        \includegraphics[width=0.75\textwidth]{TablaADD}
        \caption{Significancia para el modelo de \emph{Winter Aditivo} sobre el conjunto de datos \texttt{EJ2.SEMANAL4}}
        \label{table:b_winter_additive_significance}
      \end{table}

      \paragraph{}
      Observando la tabla, vemos que que tanto para la constante de suavizado para la tendencia como para la estacionalidad no son significativos, es decir, no se rechaza el test $\alpha_2$ y $\alpha_3$ igual a 0. Como el estimador $\alpha_2$ de es cercano a 0 y al existir estacionalidad significará que es conveniente diferenciar. Para $\alpha_3$ será que son indices estacionales deterministas.

      \begin{figure}[htb!]
        \centering
        \includegraphics[width=0.75\textwidth]{ACF_ADD}
        \caption{Gráfico de autocorrelaciones (correlograma) para los residuales del modelo de \emph{Winter Aditivo} sobre el conjunto de datos \texttt{EJ2.SEMANAL4}}
        \label{img:b_winter_additive_residuals_correlogram}
      \end{figure}

      \paragraph{}
      Siguiendo la línea de lo comentado para el ACF del seasonal, vemos que para el retardo 2 de nuevo vuelve a ser una autocorrelación muy alta que indicará que solo se validará si no encontramos uno mejor. Para las autocorrelaciones para los periodos estacionales no se observa valores altos.

      \begin{figure}[htb!]
        \centering
        \includegraphics[width=0.75\textwidth]{WN_ADD}
        \caption{Gráfico sobre el contraste de ruido blanco para los residuales del modelo de \emph{Winter Aditivo} sobre el conjunto de datos \texttt{EJ2.SEMANAL4}}
        \label{img:b_winter_additive_test_white_noise}
      \end{figure}

      \paragraph{}
      Siguiendo el análisis de este modelo para determinar si es un modelo con ruido blanco o no, vemos en dicha gráfica que de nuevo se rechaza para los primeros retardos y por tanto no será ruido blanco, algo indeseable para validar el modelo.

      \paragraph{}
      Por último analizaremos el modelo Winter multiplicativo.

    \subsection{Winter Multiplicativo}

      \begin{listing}[htb!]
        \centering
        \inputminted{SAS}{./res/code/b-01-esm-wintermul.sas}
        \caption{Ajuste de un modelo de \emph{Winter Multiplicativo} sobre el conjunto de datos \texttt{EJ2.SEMANAL4}}
        \label{code:b_winter_multiplicative_esm}
      \end{listing}

      \paragraph{}
      Para comenzar se adjunta la tabla de significancia, la cual se muestra en la \autoref{table:b_winter_multiplicative_significance}.

      \begin{table}[htb!]
        \centering
        \includegraphics[width=0.75\textwidth]{TablaMUL}
        \caption{Significancia para el modelo de \emph{Winter Multiplicativo} sobre el conjunto de datos \texttt{EJ2.SEMANAL4}}
        \label{table:b_winter_multiplicative_significance}
      \end{table}

      \paragraph{}
      Observamos en la tabla anterior, que en este caso si se rechaza para la constante de suavizado estacional con un pvalor$ <0.001$ y para el nivel. Sin embargo, para la constantes de suavizado para la tendencia , no se rechaza la hipótesis de $\alpha_2 =0$, por lo que concluimos no será adecuado suavizar dicha componente. Vemos que su estimador es próximo a cero, por lo que será conveniente diferenciar o utilizar modelos ARIMA.

      \begin{figure}[htb!]
        \centering
        \includegraphics[width=0.75\textwidth]{ACF_MUL}
        \caption{Gráfico de autocorrelaciones (correlograma) para los residuales del modelo de \emph{Winter Multiplicativo} sobre el conjunto de datos \texttt{EJ2.SEMANAL4}}
        \label{img:b_winter_multiplicative_residuals_correlogram}
      \end{figure}

      \paragraph{}
      En el gráfico adjunto anteriormente, vemos como los primeros retardos son ligeramente menores a los de los otros modelos, donde en ningún caso sobrepasan las bandas. Los retardos para los periodos no son muy altos, quizás solo la autocorrelación 18 pero no es tan influyente en el ajuste.

      \begin{figure}[htb!]
        \centering
        \includegraphics[width=0.75\textwidth]{WN_MUL}
        \caption{Gráfico sobre el contraste de ruido blanco para los residuales del modelo de \emph{Winter Multiplicativo} sobre el conjunto de datos \texttt{EJ2.SEMANAL4}}
        \label{img:b_winter_multplicative_test_white_noise}
      \end{figure}

      \paragraph{}
      Para finalizar, vemos el gráfico WN para contrastar si los residuales del modelo son ruido blanco. En este caso, diferenciandose ligeramente con los modelos anteriores, vemos que para $\alpha$ del 0.01 no se rechaza ninguno y solo 3 para $\alpha$ 0.05.

      \paragraph{}
      Por último compararemos SSE de cada modelo a modo de información adicional:

      \begin{table}[htb!]
        \centering
        \begin{tabular}{|l|r|}
            \hline
            \bfseries Modelo & $SSE$
            \csvreader[head to column names]{res/data/sse.csv}{}
            {\\\hline\MODEL & \SSE}
            \\ \hline
        \end{tabular}
        \caption{Relación entre los modelos ajustados y su \emph{Suma de Cuadrados del Error}.}
        \label{table:sse_comparative}
      \end{table}

  \section{Elegir el modelo que se considere más apropiado entre los tres del \autoref{sec:b} y con ese modelo dar las predicciones para las próximas $6$ observaciones. Justificar la elección del modelo.}
  \label{sec:c}

    \begin{listing}[htb!]
      \centering
      \inputminted{SAS}{./res/code/c-01-prediction.sas}
      \caption{Ajuste y predicción de las $5$ observaciones siguientes de un modelo de \emph{Winter Multiplicativo} sobre el conjunto de datos \texttt{EJ2.SEMANAL4}}
      \label{code:winter_multiplicative_prediction}
    \end{listing}

    \paragraph{}
    Tras realizar un exhaustivo análisis en el apartado b, de los 3 modelos de suavizado exponencial, hemos considerado que el mejor modelo es el multiplicativo de Winter.
    Esta eleccion la hemos concluido por los motivos mencionados anteriormente :

    \begin{itemize}
      \item Pvalores de la \autoref{table:b_winter_multiplicative_significance} (Constante de Suavizado Estacional Significativa)
      \item Gráfica ACF de la \autoref{img:b_winter_multiplicative_residuals_correlogram} (Autocorrelaciones menores que otro modelos)
      \item Grafica WN de la \autoref{img:b_winter_multplicative_test_white_noise} (Retardos con menor probabilidad menor de ser ruido blanco)
    \end{itemize}

    \paragraph{}
    Por último, obtenemos las predicciones para las seis siguientes observaciones, las cuales se muestran en la \autoref{table:predictions}.

    \begin{table}[htb!]
      \centering
      \includegraphics[width=0.75\textwidth]{PredC}
      \caption{Prediciones para las próximas $6$ observaciones basadas en el modelo de \emph{Winter Multiplicativo}.}
      \label{table:predictions}
    \end{table}

  \section{Utilizando en el ajuste solamente los datos hasta el final de $1990$ que no incluyan ningún caso de $1991$, calcular los errores de predicción para el año $1991$ y su correspondiente $SSE_p$ (suma de $s$ errores al cuadrado correspondientes a predicciones $\{1, 2, ..., s\}$ pasos hacia adelante) para los tres modelos del \autoref{sec:b}. Comentar si la elección hecha en el apartado \autoref{sec:c} está de acuerdo con los resultados obtenidos en este caso al comparar la capacidad de predicción de los distintos modelos para el año $1991$. Adjuntad el programa con el lenguaje control que hayáis utilizado en este apartado.}
  \label{sec:d}

    \paragraph{}
    Para calcular el $SSE_p$, primero explicaremos resumidamente que és y en que basa. El $SSE_p$ es el cálculo de una medida para comparar la capacidad de predicción de un modelo de series temporales.
    Se trata de predecir observaciones de una serie estacional de periodo s, para
    medir la capacidad de predicción de un modelo ajustado a dicha serie.

    \paragraph{}
    Si se dispone de n observaciones en total, $x_1,x_2...x_n$ se reservan las últimas k observaciones, donde k es un múltiplo de s. Para el ajuste se utilizan m observaciones ($m = n - k$) y la medida se obtiene sumando los cuadrados de los residuales $\{1, 2, ..., k\}$ pasos hacia adelante. Esto se define en la \autoref{eq:sse_p}.

    \begin{equation}
    \label{eq:sse_p}
      \begin{split}
        SSE_p
        &= \sum_{j = 1} ^ k (x_{m + j} - x_{m}(j)) ^ 2 \\
        &= \sum_{j = 1} ^ k e_m(j) ^ 2 \\
        &= e_m(1) ^ 2 + e_m(2) ^ 2 + ... + e_m(k) ^ 2
      \end{split}
    \end{equation}

    \paragraph{}
    En la \autoref{table:sse_p_complete} se muestran los valores predichos, los errores de predicción, y el $SSE_p$ acumulado para cada una de las $13$ observaciones predichas, traduciéndose finalmente en el $SSE_p$ en la observacion $130$.

    \begin{table}[htb!]
      \centering
      \includegraphics[width=\textwidth]{cabecera}
      \includegraphics[width=\textwidth]{Errores}
      \caption{Suma de Cuadrados del Error de Predicción acumulada para $13$ observaciones hacia delante para el modelo de \emph{Suavizado Exponencial con Estacionalidad}, modelo de \emph{Winter Aditivo} y modelo de \emph{Winter Multiplicativo}.}
      \label{table:sse_p_complete}
    \end{table}

    \paragraph{}
    De manera resumida, en la \autoref{table:sse_p_summary} se adjunta la capacidad de predicción para los $3$ modelos estudiados, donde observamos que el modelo con menor error de prediccion para las $13$ próximas observaciones es el modelo de \emph{Winter Multiplicativo} cuyo valor $SSE_p = 59743.29$ es el menor de todos ellos..

    \begin{table}[htb!]
      \centering
      \includegraphics[width=\textwidth]{SSEP_D}
      \caption{Suma de Cuadrados del Error de Predicción $k =13$ observaciones hacia delante para el modelo de \emph{Suavizado Exponencial con Estacionalidad}, modelo de \emph{Winter Aditivo} y modelo de \emph{Winter Multiplicativo}}
      \label{table:sse_p_summary}
    \end{table}

    \paragraph{}
    El código fuente utilizado para el cálculo de la suma de cuadrados del error de predicción $k$ pasos hacia adelante ($SSE_p$) se ha incluido en este documento. En concreto el \autoref{code:d_cross_validation} se refiere a la generación del conjunto de datos con el cual ajustar el modelo. Es decir, sin las observaciones que se van a predecir, para después comparar los resultados (esta estrategia de reserva de observaciones se conocer como \emph{Holdut}). El siguiente paso es el ajuste y predicción de las $k = 13$ observaciones siguientes para los modelos. Esto se hace en el \autoref{code:d_prediction_error_esm_seasonal}, \autoref{code:d_prediction_error_esm_winteradd}, y \autoref{code:d_prediction_error_esm_wintermul}. Finalmente, el cálculo de la \emph{Suma de Cuadrados del Error de Predicción} se lleva a cabo en el \autoref{code:d_summary_error}.

    \begin{listing}[htb!]
      \centering
      \inputminted{SAS}{./res/code/d-01-cross-validation.sas}
      \caption{Generación del conjunto de datos con el cual se ajustar el modelo.}
      \label{code:d_cross_validation}
    \end{listing}

    \begin{listing}[htb!]
      \centering
      \inputminted{SAS}{./res/code/d-02-prediction-error-esm-seasonal.sas}
      \caption{Ajuste del modelo de \emph{Suavizado Exponencial con Estacionalidad} y cálcula de las predicciones $k = 13$ pasos hacia delante.}
      \label{code:d_prediction_error_esm_seasonal}
    \end{listing}

    \begin{listing}[htb!]
      \centering
      \inputminted{SAS}{./res/code/d-02-prediction-error-esm-winteradd.sas}
      \caption{Ajuste del modelo de \emph{Winter Aditivo} y cálcula de las predicciones $k = 13$ pasos hacia delante.}
      \label{code:d_prediction_error_esm_winteradd}
    \end{listing}

    \begin{listing}[htb!]
      \centering
      \inputminted{SAS}{./res/code/d-02-prediction-error-esm-wintermul.sas}
      \caption{Ajuste del modelo de \emph{Winter Multiplicativo} y cálcula de las predicciones $k = 13$ pasos hacia delante.}
      \label{code:d_prediction_error_esm_wintermul}
    \end{listing}

    \begin{listing}[htb!]
      \centering
      \inputminted{SAS}{./res/code/d-03-error-summary.sas}
      \caption{Cálculo del la \emph{Suma de Cuadrados del Error de Predicción} $k = 13$ pasos hacia delante para todos los modelos ajustados.}
      \label{code:d_summary_error}
    \end{listing}

  \section{Obtener con el \texttt{proc forecast} de \emph{SAS} el $SSE_p$ para el modelo de \emph{Winter Multiplicativo} con las mismas constantes de suavizado y los valores iniciales de los parámetros lo más próximos posible a los obtenidos en el \autoref{sec:d} con el \texttt{proc esm} para este modelo. Adjuntar el programa con el lenguaje control que hayáis utilizado para obtenerlo.}
  \label{sec:e}

    \paragraph{}
    En la \autoref{table:sse_p_forecast_1} se muestra la suma de cuadrados del error de predicción obtenida de la misma manera que en el \autoref{sec:d}. Esto es fijando los mismos valores iniciales \texttt{ASTART} y \texttt{BSTART} referidos al término independiente y la pendiente respectivamente. En cuanto a los parámetros estacionales, se ha probado con distintos valores para el atributo \texttt{NSSTART} (número de observaciones para estimar los parámetros estaciones) comprobando que el que más se acerca es \texttt{NSSTART = 2}, obteniendo un $SSE_p = 62801.35$, muy semejante al obtenido con \textit{proc esm}. El

    \begin{table}[htb!]
      \centering
      \includegraphics[width=0.7\textwidth]{TablaSSEPFOR}
      \caption{Cálculo de la \emph{Suma de Cuadrados del Error de Predicción ($SSE_p$)} $k = 13$ pasos hacia adelante por el método básico (1) con \texttt{proc forecast} para el modelo de \emph{Winter Multiplicativo}}
      \label{table:sse_p_forecast_1}
    \end{table}

    \paragraph{}
    En el \autoref{code:prediction_forecast_method_1} se incluyen las sentencias utilizadas para el cálculo de la suma de cuadrados de predicción $k = 13$ pasos hacia adelante con el \texttt{proc forecast}.

    \begin{listing}[htb!]
      \centering
      \inputminted{SAS}{./res/code/e-prediction-forecast-method-1.sas}
      \caption{Cálculo de la \emph{Suma de Cuadrados del Error de Predicción ($SSE_p$)} $k = 13$ pasos hacia adelante por el método básico (1) con \texttt{proc forecast} para el modelo de \emph{Winter Multiplicativo}}
      \label{code:prediction_forecast_method_1}
    \end{listing}

    \paragraph{}
    Paralelamente a la metodología indicada en el \autoref{code:prediction_forecast_method_1}, se puede seguir otra diferente para llegar a los mismos resultados que elimina la labor de probar diferentes valores para el parámetro \texttt{NSSTART}. Esta se incluye en el \autoref{code:prediction_forecast_method_2}. Los resultados que se obtienen siguiendo estrategia se muestran en la \autoref{table:sse_p_forecast_2}. Como vemos en la última fila, los resultados son equivalentes.

    \begin{table}[htb!]
      \centering
      \includegraphics[width=0.5\textwidth]{cabecera2}
      \includegraphics[width=0.5\textwidth]{SSEPFOR}
      \caption{Cálculo de la \emph{Suma de Cuadrados del Error de Predicción ($SSE_p$)} $k = 13$ pasos hacia adelante por el método alternativo (2) con \texttt{proc forecast} para el modelo de \emph{Winter Multiplicativo}}
      \label{table:sse_p_forecast_2}
    \end{table}

    \begin{listing}[htb!]
      \centering
      \inputminted{SAS}{./res/code/e-prediction-forecast-method-2.sas}
      \caption{Cálculo del error de predicción $SSE_p$ mediante el \texttt{proc forecast} por el método alternativo (2).}
      \label{code:prediction_forecast_method_2}
    \end{listing}

  \section{Ajustar un modelo para la serie $\{ Xt \}$ con el módulo \texttt{Time Series Forecasting System} de \emph{SAS} razonando porqué se ha elegido. Utilizar el modelo elegido para predecir valores futuros de esta serie y establecer la comparación con los seis valores obtenidos en el \autoref{sec:c} junto con sus bandas de predicción.}
  \label{sec:f}


    \paragraph{}
    Para decidir que modelo utilizar hemos comparado el multiplicativo con el modelo estacional, realizando el mismo proceso que en el \autoref{sec:c}. Los resultados no han sido concluyentes porque habia una similitud muy alta entre ambos. En el modelo estacional eran significativas la constante que suaviza el nivel y la que suaviza la componente estacional, y en el de Winter multiplicativo ocurria igual, y la que suaviza la tendencia no era significativa y por tanto no útil en el modelo. La unica minima diferencia es el $SSE$, que resultaba menor para el multiplicativo, siendo $417937.2$ frente a $417943.1$ del estacional.

    \begin{figure}[htb!]
      \centering
      \includegraphics[width=\textwidth]{semanal-residuals}
      \caption{Gráfico de la serie, gráfico de dispersión \emph{rango-media}, \emph{correlograma} y \emph{periodograma} de la serie generada por los residuales del modelo de \emph{Winter Multiplicativo} ajustado al conjunto de datos \texttt{EJ2.SEMANAL}}
      \label{img:residuals_semanal}
    \end{figure}

    \paragraph{}
    A continuación se adjunta la tabla de predicciones semanales para el modelo de \emph{Winter Multiplicativo}, obtenida con el siguiente código \texttt{SAS} del \autoref{code:f_export}.

    \begin{listing}[htb!]
      \centering
      \inputminted{SAS}{./res/code/f-export.sas}
      \caption{Conjunto de datos donde se almacenan los resultados del modelo de \emph{Winter Multiplicativo}}
      \label{code:f_export}
    \end{listing}

    \paragraph{}
    Los datos que se muestran en la \ref{table:pred_semanal} son meramente informativos. Lo interesante de los mismos es la suma de estas predicciones agrupadas en grupos de $4$ en cuatro. Esto permite para comprarlas con lo obtenido en el \autoref{sec:c} para la serie temporal de datos agrupados cada $4$ semanas. Esto se ha llevado a cabo mediante la ejecución del \autoref{code:f_comparison}.

    \begin{table}[htb!]
      \centering
      \includegraphics[width=0.5\textwidth]{Tabla1PredFsemanal}
      \includegraphics[width=0.5\textwidth]{Tabla2PredFsemanal}
      \caption{Predicciones del modelo de \emph{Winter Multiplicativo} para la serie temporal referida a observaciones semanales.}
      \label{table:pred_semanal}
    \end{table}

    \paragraph{}
    A continuación se muestran la tabla que obtuvimos en el \autoref{sec:c} y en la recientemente obtenida sumando las predicciones semanales. Se puede observar como todas las predicciones obtenidas de la suma entran en la banda de prediccion de la serie cada $4$ semanas. Esto indica un buen ajuste de prediccion con este modelo.

    \paragraph{}
    Además, relizamos los mismos pasos parar el modelo estacional, y obtenemos que la primera predicción de la suma se sale de la banda de prediccion ($363.4203$), lo cual es otro argumento para seleccionar el modelo multiplicativo frente al estacional ya que este ajusta peor las predicciones.

    \begin{table}[htb!]
      \centering
      \includegraphics[width=0.75\textwidth]{TablaSem4PredC}
      \caption{Predicciones obtenidas por el modelo de \emph{Winter Multiplicativo}, agrupadas en $4$ conjuntos de semanas, para poder ser comparadas con los resultados del \autoref{sec:c}. Además, estas incluyen sus respectivas bandas de predicción.}
      \label{table:pred_semanal_4}
    \end{table}

    \begin{table}[htb!]
      \centering
      \includegraphics[width=0.4\textwidth]{TablaSuma4SemanalPredF}
      \caption{Predicciones agrupadas en Semanas de 4 para comparar con \autoref{sec:c}.}
      \label{table:pred_grouped_semanal}
    \end{table}

    \begin{listing}[htb!]
      \centering
      \inputminted{SAS}{./res/code/f-comparison.sas}
      \caption{Comparación de las predicciones obtenidas de manera semanal frente a las obtenidas por grupos de 4 semanas.}
      \label{code:f_comparison}
    \end{listing}

  \appendix
  \renewcommand{\thesection}{(\Roman{section})}
  \renewcommand\thesubsection{(\Roman{section}.\Roman{subsection})}

  \section{Código Fuente Auxiliar}
  \label{appendix:source_code}

    \subsection{Código Fuente \texttt{SAS}}
    \label{appendix:source_code_sas}

      \begin{listing}[H]
        \centering
        \inputminted{SAS}{./res/code/a-01-data.sas}
        \caption{Generación del conjunto de datos \texttt{EJ2.SEMANAL}}
        \label{code:data_import}
      \end{listing}


    \subsection{Código Fuente \texttt{R}}
    \label{appendix:source_code_r}

      \begin{listing}[H]
        \centering
        \inputminted{R}{./res/code/r/functions.r}
        \caption{Generación de cojunto de gráficos (gráfico de la serie, gráfico \emph{rango-media}, \emph{correlograma} y \emph{periodograma}) para análisis descriptivo de las series \texttt{SEMANAL}, \texttt{SEMANAL4} y \texttt{SEMANALRESIDUOS}.}
        \label{code:semanal_analysis}
      \end{listing}


\end{document}
